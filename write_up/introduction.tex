\section{Introduction}
\label{sec:introduction}

The political science literature on Foreign Direct Investment (FDI) has focused largely on how politics shapes the flow of FDI across countries. The central insight of this literature is that multinational corporations (MNCs) face an ``obsolescing bargain'' against the host government. Once the MNC has sunk its investment, it is vulnerable to the host government's changing regulations, backtracking on deals, or even expropriating its properties \citep{Li2009a, Sawant2010}. Certain institutional and political characteristics, such as numerous veto players, executive constraint, or strong property rights, allow the host government to make a credible commitment and thus ameliorate the severity of the ``obsolescing bargain'' problem \citep{Busse2007, Jensen2014, Li2003}. According to the literature, MNCs should invest more in countries with these characteristics.

This dominant approach in the literature has three long-standing issues that my paper will address. First, the majority of the literature relies on FDI stock and flow data as the outcome of interest even though they are often not an appropriate measure for the scale of MNCs' activities \citep{Kerner2014}. While it would be ideal to use firm-level data instead, both the lack of cross-national firm-level data and a suitable statistical model have posed a challenge.

Second, while there has been much focus on MNCs choosing host countries, the literature has largely neglected the other side of the investment decision: what are countries' preferences regarding MNCs? Consider the established finding that democracies receive more FDI. Without controlling for countries' preferences, it is difficult to interpret this fact as democracies actively pursuing MNCs or as MNCs finding democracies attractive. Not only are countries' preferences central to the modeling of investment decision, arguably it is also more steeped with politics and deserves more attention. \citet{Pinto2013} and \citet{Pandya2016} are two pioneering works in this area of research, proposing partisan politics and regime types as factors shaping countries' preferences for FDI. However, while their theories are ground-breaking, the empirical estimation of countries' preferences remains difficult.

Third, in addition to empirical issues raised above, I propose that we need to theorize about countries' preferences for FDI quality. While the political science literature has largely focused on the quantity of FDI, national policies and discourses pay much attention to the quality of FDI, using various incentives and restrictions to target certain types of FDI. Indeed, MNCs come with varying capital, demand for labor, and technology, all of which have different effects on the host country's economy. For example, policy makers and scholars have highlighted high-tech MNCs as a source of technological transfer for developing host countries, allowing them to upgrade their technical capacity and improve their productivity \citep{Findlay1978, Nunnenkamp2004}. While such high-quality FDI has been enthusiastically endorsed by the development community, I argue that only governments with a long time horizon want to attract high-tech FDI because technological transfer takes time to pay off.

In sum, the current literature would benefit from an analysis that is capable of using firm-level data to estimate both firms' and countries' preferences for each other's characteristics. To accomplish this goal, I adapt the two-sided matching model originally designed for the labor market and the marriage market. In this model, both firms and countries evaluate their available options and choose the best according to their utility functions. As in many social science contexts, we only observe the final firm-country matches and not the full set of available options (also known as the opportunity set). I solve this problem by using the Metropolis Hastings algorithm, a Markov chain Monte Carlo (MCMC) approach that repeatedly samples new opportunity sets and rejects them at an appropriate rate to approximate their true distribution. Since the two-sided matching model is derived explicitly from actors' utility functions, their parameters also enjoy a straightforward interpretation in the utility space instead of some aggregate outcomes.

The paper proceeds as follows. \Cref{sec:literature_issues} discusses the three long-standing issues with the literature and how they can be improved. \Cref{sec:model} lays out the utility structure in the two-sided matching model and describes the matching process. \Cref{sec:tsl_estimate} shows how the model can be estimated with the Metropolis-Hastings algorithm. \Cref{sec:application} shows an application of the model on a census of Japanese firms overseas. \Cref{sec:result} presents the result. \Cref{sec:conclusion} concludes.