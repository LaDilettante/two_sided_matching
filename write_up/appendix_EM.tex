\section{EM algorithm}

Our data contains a random sample of firms and the countries in which they invest. We want to find the parameters that maximize the likelihood of this observed data. This likelihood is:

\[
L = \prod_{i,j: \text{i is matched with j}} Pr(A_{ij})
\]

where $Pr(A_{ij})$ is the probability of a specific match between firm $i$ and country $j$. $Pr(A_{ij})$ can be calculated as follows:

\begin{align}
&Pr(A_{ij}) \\
&= \sum_{k=1}^R Pr(A_{ij}|S_{ik}) Pr(S_{ik}) \\
&= \sum_{k=1}^R Pr(A_{ij} | S_{ik}) \prod_{m \in O_k} Pr(o_{im} = 1) \prod_{n \in \bar O_k} Pr(o_{in} = 0) \\
&= \sum_{k:j \in O_k} \frac{\exp(\alpha w_{ij})}{\displaystyle\sum_{h \in O_k} \exp(\alpha w_{ih})} \prod_{m \in O_k, m > 0} \frac{\exp(\beta x_{i})}{1 + \exp(\beta x_i)} \\
&\times \prod_{n \in \bar O_k, n > 0} \frac{1}{1 + \exp(\beta x_i)}
\end{align}

Here, the term $Pr(o_{ij} = 1)$ represents the probability that country $j$ makes an offer to firm $i$, through which the official's preference enters our estimation. On the other side, $Pr(A_{ij}) | S_{ik}$ is the probability that firm $i$ will accept the offer from official $j$, given the offering set $O_k$. The firm's preference is reflected in our estimation through this term, $Pr(A_{ij})|S_{ik}$.\footnote{The appendix shows how these terms are derived.}

It is important to note that the offering set $O_k$ contains \textit{all} offers that firm $i$ receives, only one of which is the observed match between firm $i$ and country $j$. The intuition is that if we observe the full set of offers that all officials make to all firms, then by looking at the final match we can see how firms and officials reject inferior offers and thus deduce their preferences.
