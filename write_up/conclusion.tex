\section{Conclusion}
\label{sec:conclusion}

In this paper, I propose the two-sided matching model to estimate firms' and countries' preferences, solving three persistent issues in the literature of FDI's political determinants. The results indicate that, for Japanese MNCs, only a country's level of development matters and not its market size, labor quality, or regime type. This finding suggests that we should take a closer look at the relationship between democracies and MNCs. Since previous works in the literature have not controlled for countries' preferences, they may have mistaken democracies' love for FDI as FDI's fondness for democracies.

On the other hand, the model's estimation of countries' preference remains lacking. Since each country has its own set of parameters, the parameter space seems too large for the current implementation of the Metropolis-Hastings algorithm to fully explore. Several solutions are possible. First, we can collapse countries into categories of interest, e.g. regime types, (categorical) time horizon length. Second, we can build a hierarchical model, modeling countries' preferences as draws from a common distribution. Such model will allow us to pool information across countries and reduce the parameter space.